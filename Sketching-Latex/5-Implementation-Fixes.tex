\section{Implementation Note: Global Subspace Estimation}

To robustly estimate the parameters, we implemented a **Global SVD** approach. This differs from the standard derivation, which typically suggests performing subspace estimation independently for each direction.

\subsection{Motivation: The Basis Ambiguity Problem}
In the standard element-wise ESPRIT approach, one would compute the Singular Value Decomposition (SVD) for each measurement matrix $Z_l$ individually to find a signal subspace $U_l$. However, the signal subspace is only unique up to a unitary rotation. If $U_l$ is a valid basis for the signal subspace of direction $l$, then $U_l Q$ is also a valid basis for any unitary matrix $Q$.

When these subspaces are estimated independently, each direction $l$ effectively chooses a \textit{different, arbitrary} rotation $Q_l$. Consequently, the Rotational Invariance Matrices ($\Psi_l$) derived from these subspaces are expressed in different coordinate systems. This makes it impossible to find a single common diagonalizing matrix $T$ in the Joint ESPRIT step, as the eigenvectors of $\Psi_l$ and $\Psi_k$ are unrelated.

\subsection{The Global SVD Algorithm}
To solve this, we enforce a single common coordinate system by estimating a \textbf{global signal subspace} from all data simultaneously.

\subsubsection{1. Construction of Global Data Matrix}
We interpret the data from all $M$ directions as being generated by a single large "virtual array" observing the same $S$ source snapshots. We vertically stack the $M$ measurement matrices $Z_l \in \mathbb{C}^{N \times S}$ into a single global data matrix $X_{\text{glob}} \in \mathbb{C}^{MN \times S}$:

\[
X_{\text{glob}} = \begin{bmatrix} Z_1 \\ Z_2 \\ \vdots \\ Z_M \end{bmatrix}
\]

\subsubsection{2. Global Subspace Extraction}
We perform a single Truncated SVD on this global matrix:
\[
X_{\text{glob}} \approx U_{\text{glob}} \Sigma_{\text{glob}} V_{\text{glob}}^H
\]
where $U_{\text{glob}} \in \mathbb{C}^{MN \times r}$ contains the $r$ dominant left singular vectors. The columns of $U_{\text{glob}}$ span the \textbf{common signal subspace} shared by all directions. Crucially, this fixes the basis rotation for the entire dataset.

\subsubsection{3. Extraction of Directional Subspaces}
The global subspace matrix $U_{\text{glob}}$ is composed of $M$ stacked blocks, each of size $N \times r$. We partition it back into blocks corresponding to each direction:
\[
U_{\text{glob}} = \begin{bmatrix} \hat{U}_1 \\ \hat{U}_2 \\ \vdots \\ \hat{U}_M \end{bmatrix} \quad \text{where } \hat{U}_l \in \mathbb{C}^{N \times r}
\]
Here, $\hat{U}_l$ represents the signal subspace for direction $l$, but unlike the independently estimated $U_l$, these $\hat{U}_l$ blocks are \textbf{locked to the same global coordinates}. They are coherent projections of the single global signal structure.

\subsubsection{4. Computation of Coherent RIMs}
For each block $\hat{U}_l$, we proceed with the standard ESPRIT invariance equation. We form $\hat{U}_{l, \uparrow}$ (first $N-1$ rows) and $\hat{U}_{l, \downarrow}$ (last $N-1$ rows) and solve:
\[
\hat{U}_{l, \uparrow} \Psi_l \approx \hat{U}_{l, \downarrow}
\]
Because all $\hat{U}_l$ blocks were derived from the same $U_{\text{glob}}$, the resulting matrices $\Psi_l$ are guaranteed to share the same set of eigenvectors (the columns of the global mixing matrix inverse). This satisfies the theoretical requirement for step 4 (Joint Diagonalization), allowing the algorithm to correctly pair and recover the $d$-dimensional frequencies.
