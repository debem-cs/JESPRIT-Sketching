\section{Results}

\subsection{Phase Unwrapping Impact}
In the theoretical derivation of ESPRIT-based methods, phase unwrapping is often cited as a necessary step to resolve the ambiguity of the frequency estimates when the phase arguments exceed the $(-\pi, \pi]$ range. However, we found that applying phase unwrapping in the JESPRIT context actually hurts performance.

Figure \ref{fig:unwrapping_comparison} compares the parameter estimation error with and without phase unwrapping enabled. It can be observed that the unwrapped version consistently yields higher error rates and is robust over a smaller range of grid scale values ($\Delta$).

\begin{figure}[h]
    \centering
    % \includegraphics[width=0.8\textwidth]{path/to/unwrapping_comparison.png}
    \fbox{\parbox{0.8\textwidth}{
        \centering
        \vspace{2cm}
        [Placeholder: Comparison of Error vs $\Delta$ with and without Phase Unwrapping] 
        \vspace{2cm}
    }}
    \caption{Comparison of estimation error with and without phase unwrapping. The unwrapped version (red) shows higher error and instability compared to the wrapped version (blue).}
    \label{fig:unwrapping_comparison}
\end{figure}

\subsection{Optimal Grid Scale ($\Delta$)}
The choice of the grid scale parameter $\Delta$ is critical for the performance of the algorithm. Experiments demonstrate that there is a distinct "sweet spot" for this parameter.

Figure \ref{fig:error_vs_delta} illustrates the estimation error as a function of the grid scale $\Delta$. The error plummets significantly when $\Delta \approx 1/\max(A)$, where $\max(A)$ corresponds to the maximum expected rate in the system. 

\begin{figure}[h]
    \centering
    % \includegraphics[width=0.8\textwidth]{path/to/error_vs_delta.png}
    \fbox{\parbox{0.8\textwidth}{
        \centering
        \vspace{2cm}
        [Placeholder: Error vs $\Delta$ showing minimum at $1/\max(A)$]
        \vspace{2cm}
    }}
    \caption{Estimation error as a function of grid scale $\Delta$. The global minimum occurs near $\Delta = 1/\max(A)$.}
    \label{fig:error_vs_delta}
\end{figure}

\subsection{Sample Complexity and Rate Range}
Finally, we analyze the sample complexity of JESPRIT—specifically, how many samples are required to successfully recover the latent factors as the problem dimensions grow. We also investigate the impact of the dynamic range of the rates in $A$.

We compared the sample complexity for two scenarios:
\begin{enumerate}
    \item \textbf{Small Range}: Rates $A$ drawn from $[0, 100]$.
    \item \textbf{Large Range}: Rates $A$ drawn from $[0, 10000]$.
\end{enumerate}

Figures \ref{fig:sample_complexity_100} and \ref{fig:sample_complexity_10000} show the success rate heatmaps for varying dimensions $d$ and ranks $r$. The results indicate that a larger range of values in $A$ improves the recoverability of the latent factors. With a larger dynamic range, the "directions" in the count space are more distinct, essentially providing a higher effective signal-to-noise ratio for the subspace estimation. This allows the algorithm to correctly discover more latent factors (higher $r$) for a given sample size compared to the small range scenario.

Furthermore, the results suggest that the sample complexity depends primarily on the number of latent factors $r$, rather than the ambient dimension $d$. As observed in the heatmaps, increasing $d$ while keeping $r$ constant results in a negligible increase in the sample size required for successful recovery.

\begin{figure}[h]
    \centering
    \begin{minipage}{0.48\textwidth}
        \centering
        % \includegraphics[width=\textwidth]{path/to/heatmap_100.png}
        \fbox{\parbox{\textwidth}{
            \centering
            \vspace{1.5cm}
            [Placeholder: Heatmap for $A \in {[0, 100]}$]
            \vspace{1.5cm}
        }}
        \caption{Sample Complexity for $A \in {[0, 100]}$.}
        \label{fig:sample_complexity_100}
    \end{minipage}
    \hfill
    \begin{minipage}{0.48\textwidth}
        \centering
        % \includegraphics[width=\textwidth]{path/to/heatmap_10000.png}
        \fbox{\parbox{\textwidth}{
            \centering
            \vspace{1.5cm}
            [Placeholder: Heatmap for $A \in {[0, 10000]}$]
            \vspace{1.5cm}
        }}
        \caption{Sample Complexity for $A \in {[0, 10000]}$.}
        \label{fig:sample_complexity_10000}
    \end{minipage}
    \caption{Comparison of Sample Complexity for different rate ranges. The larger range (right) allows for successful recovery of higher ranks.}
\end{figure}
